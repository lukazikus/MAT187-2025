\begin{enumerate}
	\item 
    The question is asking for one time when $\frac{dx}{dt}<0$ and $\frac{dy}{dt}>0$.
    
        Note that this solution provides more information than what is asked of the students. They only need to find one moment in time.
    
        The particle is moving left when $\frac{dx}{dt} < 0$, and it is moving right when $\frac{dx}{dt} > 0$. Similarly, the particle is moving up when $\frac{dy}{dt} > 0$, and it is moving down when $\frac{dy}{dt} < 0$. 
        
        For moving left, we have:
\[
    \begin{aligned}
       4\cos{4t} < 0\\
            \implies \cos{4t} < 0\\
            \implies \frac{\pi}{2} < 4t < \frac{3\pi}{2}, \frac{5\pi}{2} < 4t < \frac{7\pi}{2}, \frac{9\pi}{2} < 4t < \frac{11\pi}{2}, \frac{13\pi}{2} < 4t < \frac{15\pi}{2}\\
            \implies \frac{\pi}{8} < t < \frac{3\pi}{8}, \frac{5\pi}{8} < t < \frac{7\pi}{8}, \frac{9\pi}{8} < t < \frac{11\pi}{8}, \frac{13\pi}{8} < t < \frac{15\pi}{8}
    \end{aligned}
\]
        
        For moving right, we can simply take the complementary times:
        \[
            0 < t < \frac{\pi}{8}, \frac{3\pi}{8} < t < \frac{5\pi}{8}, \frac{7\pi}{8} < t < \frac{9\pi}{8}, \frac{11\pi}{8} < t < \frac{13\pi}{8}, \frac{15\pi}{8} < t < 2\pi
        \]
        For moving down, we can write a very similar set of inequalities, except now $3t$ is the main argument instead of $4t$ and we exclude one of the inequalities to stay within the required bound for $t$:
        
        \[
        \begin{aligned}
            &3\cos{3t} < 0\\
            \implies &\cos{3t} < 0\\
            \implies &\frac{\pi}{2} < 3t < \frac{3\pi}{2}, \frac{5\pi}{2} < 3t < \frac{7\pi}{2}, \frac{9\pi}{2} < 3t < \frac{11\pi}{2}, \frac{13\pi}{2} < 3t < \frac{15\pi}{2}\\
            \implies &\frac{\pi}{6} < t < \frac{\pi}{2}, \frac{5\pi}{6} < t < \frac{7\pi}{6}, \frac{3\pi}{2} < t < \frac{11\pi}{6}
        \end{aligned}
        \]
        Like before, we determine when the particle is moving up by finding the complementary times:
        \[
            0 < t < \frac{\pi}{6}, \frac{\pi}{2} < t < \frac{5\pi}{6}, \frac{7\pi}{6} < t < \frac{3\pi}{2}, \frac{11\pi}{6} < t < 2\pi
        \]
        
        To determine a time $t$ where the particle moves left and up, we can take the intersection between the intervals found for moving left and moving up:
        
        \[
            \frac{\pi}{8} < t < \frac{\pi}{6}, \frac{5\pi}{8} < t < \frac{5\pi}{6}, \frac{7\pi}{6} < t < \frac{11\pi}{8}, \frac{11\pi}{6} < t < \frac{15\pi}{8}
        \]

    \item 
    \begin{enumerate}
        \item All plots are circles with radius 1 centered at the origin, except for set 4 which is just the line segment $y=x$ from $(0,0)$ to $(1,1)$ and set 5 which is a semi-circle with radius 1 defined for $y > 0$. The direction for the circles is counterclockwise except set 3 which is clockwise. For the line, the direction is left-down since $t=0$ corresponds to the coordinate $(1,1)$ and $t=\pi$ corresponds to $(0,0)$.

        \item All sets except the 4th and 5th ones are equivalent to each other. Their shapes are circles with radius 1 centered at the origin.
    \end{enumerate}

    \item
    We first plot the functions $x=\cos{t}, y=\sin{3t}$. Tracing $0 < t < 2\pi$, we can plot the parametric equation on the $x-y$ plane.
	
	    \begin{figure}[!ht]
	    \centering
    
        \begin{tikzpicture}[line cap=round,line join=round,>=triangle 45,x=1cm,y=1cm]
            \begin{axis}[
                x=2cm,y=2cm,
                axis lines=middle,
                ymajorgrids=true,
                xmajorgrids=true,
                xmin=-0.22083994729243386,
                xmax=6.549381871767769,
                ymin=-2.164789358246741,
                ymax=2.0599704714017775,
            ]
            \clip(-0.22083994729243386,-2.164789358246741) rectangle (6.549381871767769,2.0599704714017775);
            \draw[line width=2pt,color=black,smooth,samples=100,domain=-0.22083994729243386:6.549381871767769] plot(\x,{cos(((\x))*180/pi)});
            \draw[line width=2pt,color=blue,smooth,samples=100,domain=-0.22083994729243386:6.549381871767769] plot(\x,{sin((3*(\x))*180/pi)});
            \begin{scriptsize}
                \draw[color=black] (6, 1.1) node {$x=\cos{t}$};
                \draw[color=blue] (6,0.5) node {$y=\sin{3t}$};
            \end{scriptsize}
            \end{axis}
        \end{tikzpicture}
    
        \end{figure}

        See \href{https://www.desmos.com/calculator/xdoroqqzdz}{here} for an interactive Desmos plot of the paths that the ants take. Feel free to play around with the parameter $t$!
        
        To find the coordinates of the horizontal and vertical tangents, we can set $\frac{dx}{dt}=-\sin{t}=0$ and $\frac{dy}{dt}=3\cos{3t}=0$, respectively. This corresponds to $t=0, \pi$ and $t=\frac{\pi}{6}, \frac{\pi}{2}, \frac{5\pi}{6}, \frac{7\pi}{6}, \frac{3\pi}{2}, \frac{11\pi}{6}$, respectively. If we plug these values of $t$ into the parametric equation, we find that the vertical tangents are located at (1,0) and (-1,0). The horizontal tangents are located at $\left(\frac{\sqrt{3}}{2},1\right), (0,-1), \left(-\frac{\sqrt{3}}{2},1\right), \left(-\frac{\sqrt{3}}{2},-1\right), (0,1), \left(\frac{\sqrt{3}}{2},-1\right)$.
        

\end{enumerate}