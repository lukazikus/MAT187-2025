\documentclass[red]{tutorial}
\usepackage[no-math]{fontspec}
\usepackage{xpatch}
	\renewcommand{\ttdefault}{ul9}
	\xpatchcmd{\ttfamily}{\selectfont}{\fontencoding{T1}\selectfont}{}{}
	\DeclareTextCommand{\nobreakspace}{T1}{\leavevmode\nobreak\ }
\usepackage{polyglossia} % English please
	\setdefaultlanguage[variant=us]{english}
%\usepackage[charter,cal=cmcal]{mathdesign} %different font
%\usepackage{avant}
\usepackage{microtype} % Less badboxes


\usepackage[charter,cal=cmcal]{mathdesign} %different font
%\usepackage{euler}

\definecolor{tolBlue}{HTML}{0077BB}
\definecolor{tolCyan}{HTML}{33BBEE}
\definecolor{tolTeal}{HTML}{009988} 
\definecolor{tolOrange}{HTML}{EE7733} 
\definecolor{tolRed}{HTML}{CC3311} 
\definecolor{tolMagenta}{HTML}{EE3377} 
\definecolor{tolGrey}{HTML}{BBBBBB}
\usepackage{tikz}
\usepackage{pgfplots}
\usetikzlibrary{arrows.meta}

\usepackage{blindtext}
\usepackage{calc, ifthen, xparse, xspace}
\usepackage{makeidx}
\usepackage[hidelinks, urlcolor=blue]{hyperref}   % Internal hyperlinks
\usepackage{mathtools} % replaces amsmath
\usepackage{bbm} %lower case blackboard font
\usepackage{amsthm, bm}
\usepackage{thmtools} % be able to repeat a theorem
\usepackage{thm-restate}
\usepackage{graphicx}
\usepackage{xcolor}
\usepackage{multicol}
\usepackage{fnpct} % fancy footnote spacing

 
\newcommand{\xh}{{{\mathbf e}_1}}
\newcommand{\yh}{{{\mathbf e}_2}}
\newcommand{\zh}{{{\mathbf e}_3}}
\newcommand{\R}{\mathbb{R}}
\newcommand{\Z}{\mathbb{Z}}
\newcommand{\N}{\mathbb{N}}
\newcommand{\proj}{\mathrm{proj}}
\newcommand{\Proj}{\mathrm{proj}}
\newcommand{\Perp}{\mathrm{perp}}
\newcommand{\Span}{\mathrm{span}\,}
\newcommand{\Img}{\mathrm{img}\,}
\newcommand{\Null}{\mathrm{null}\,}
\newcommand{\Range}{\mathrm{range}\,}
\newcommand{\rref}{\mathrm{rref}}
\newcommand{\Rank}{\mathrm{rank}}
\newcommand{\nnul}{\mathrm{nullity}}
\newcommand{\mat}[1]{\begin{bmatrix}#1\end{bmatrix}}
\renewcommand{\d}{\mathrm{d}}
\newcommand{\Id}{\operatorname{id}}


\theoremstyle{definition}
\newtheorem{example}{Example}[section]
\newtheorem{defn}{Definition}[section]

%\theoremstyle{theorem}
\newtheorem{thm}{Theorem}[section]

\pgfkeys{/tutorial,
	name={Tutorial 4},
	author={},
	course={MAT 187},
	date={},
	term={},
	title={Integration Techniques}
	}

\begin{document}
	\begin{tutorial}
			\begin{objectives}
        In this tutorial you practice manipulating integrals.
	\end{objectives}

		\vspace{-.5em}
		\subsection*{Problems}
		\vspace{-.5em}


\begin{enumerate}
    \item This question is about integration by parts. For each integral, identify what the ``parts'' are (i.e., the $u$ and $v$ in the formula
    $\displaystyle \int u'v = uv-\int uv'$) and then evaluate them.
    \begin{enumerate}
        \item $\displaystyle \int x\sin x\,\mathrm d x$
        \item $\displaystyle\int\log_b x\,\mathrm d x$ where $b>0$.
    \end{enumerate}
    \item 
    For an unknown function $f$, you know $\displaystyle\int_1^2 xf'(2x)\,\mathrm d x=8$ and 
    	\begin{align*}
    	    \hfill\begin{tabular}{c||c|c|c|c|c|c|c}
            $x$&$0$&$1$&$2$&$3$&$4$&$5$&$6$\\
            \hline
            $f(x)$&$2$&$-4$&$10$&$6$&$-2$&$8$&$-12$
            \end{tabular}\hfill\null
    	\end{align*}

        Use this information to find $\displaystyle\int_1^2 f(2x)\,\mathrm d x$.

    \item \phantom{x}
    \begin{enumerate}

        %\item Find the integral $\displaystyle\int\frac{dx}{x^2-2x+10}$.
        \item \phantom{x}
        
        \begin{minipage}{0.5\textwidth}
        The upper half of an ellipse centered at the origin with axes $a$ and $b$ is described by $y = \frac{b}{a}\sqrt{a^2-x^2}$ (see figure). Use an integral to find the area of the ellipse in terms of $a$ and $b$.

        \emph{Hint: you may want to try trigonometric substitution.}
        \end{minipage}%
        \begin{minipage}{0.5\textwidth}
            \begin{center}
                \begin{tikzpicture}[scale=0.45,line width=1]
                    \draw [tolTeal] (0,0) ellipse (3 and 2);
                        \draw[->] (-4,0) -- (4.2,0) node [below] {$x$};
                    \draw[->] (0,-2.5) -- (0,3.2) node [left] {$y$};
                    \node at (3,0) [below right,tolTeal] {$a$};
                    \node at (0,2) [above left,tolTeal] {$b$};
                \end{tikzpicture}
            \end{center}
    	\end{minipage}

        \item Evaluate $\int \frac{1}{x^2 - 9}\,\mathrm d x$ for $x > 3$ using trigonometric substitution.
        
        \item \phantom{x}
        
        \begin{minipage}{0.5\textwidth}
        A ball with radius 1 is placed inside a cone that has a vertical slope of 1. Set up an integral to determine the cross sectional area of the region underneath the ball but within the cone (grey in the figure).
        Then, use trigonometric substitution to evaluate the integral.
        
        \textit{Hint: Use the fact that the ball must be tangent to the cone.}

        \end{minipage}%
        \begin{minipage}{0.5\textwidth}
            \begin{center}
                \begin{tikzpicture}[scale=0.75,line width=1]
                    \fill[gray] (0,0) -- ({1/sqrt(2)},{1/sqrt(2)}) arc (-45:-135:1) --cycle;
                    \draw [tolMagenta] (0,{sqrt(2)}) circle (1);
                    \draw [tolTeal] (-2,2) -- (0,0) -- (2,2);
                    \draw[->] (-2,0) -- (2,0) node [below] {$x$};
                    \draw[->] (0,-0.5) -- (0,3) node [left] {$y$};
                \end{tikzpicture}
            \end{center}
    	\end{minipage}
    \end{enumerate}


    \newpage
    \item The Gamma function is defined by the formula 

    \[
        \Gamma(x) = \int_0^\infty e^{-t}t^{x-1}\,\mathrm dt
    \]
    It's an extension of the factorial function that works for non-integer values. If you ever graph $x!$ in desmos, this is what's being graphed!

    In this question, we will study the related antiderivative
    \[
        \gamma_b(n) = \int_0^b e^{-t}t^{n-1}\,\mathrm dt.
    \]
    

    Where $n$ is an integer greater than one.
    \begin{enumerate}
        \item Graph $x!$ in Desmos.
    
        \item Use integration by parts to find a simplified expression for:

        \[
            \gamma(n) - (n-1)\gamma(n-1) 
        \]

        \emph{Hint: Apply one integration by parts to $\gamma(n)$}
        \item Using the expression in part b), show that

        \[\lim_{b\to \infty} \gamma(n) - (n-1) \gamma(n-1) =0\]

        Provided $n\geq 1$.
        
        \item Assuming $\lim_{b\to \infty} \gamma(1) = 1$, can you determine what $\lim_{b\to \infty} \gamma(4)$ is?

        
        \item With the same assumptions, can you find a formula for $\lim_{b\to \infty}\gamma(n)$?
        \item Does your recursive formula for $\lim_{b\to \infty}\gamma$ still hold
        when computing $\lim_{b\to\infty}\gamma(x)$ when $x\in \R$ and $x\geq 1$? Explain.
    \end{enumerate}    
\end{enumerate}


















	\end{tutorial}

	\begin{solutions}
		\begin{enumerate}
\item 
\begin{enumerate}
    \item Choosing $u=x, dv=\sin{x}\mathrm dx$, the integral becomes 
    \[
    -x\cos{x} - \int-\cos{x}\mathrm dx
    =-x\cos{x} + \sin{x} + C
    \]

    \item Using integration by parts, we can let $u = \log_b x, \mathrm dv = \mathrm dx$, giving $\mathrm du = \frac{dx}{x\ln b}, v = x$. Then we get
		
		\[
			\int\log_b x \mathrm dx \\
			= x\log_b x - \int\frac{x\mathrm dx}{x\ln b} \\
			= x\log_b x - \frac{x}{\ln b} + C \\
			= \frac{1}{\ln b}(x\ln x - x) + C
		\]
\end{enumerate}
\item
Using integration by parts, we can let $u = f(2x), \mathrm dv = \mathrm dx$, giving $\mathrm du = 2f'(2x)\mathrm dx, v = x$. Applying the integration by parts formula gives:
        
        \[
            \int_1^2 f(2x)\mathrm dx \\
            = \left[xf(2x)\right]_1^2 - \int_1^2 2xf'(2x)\mathrm dx \\
            = (2)(-2) - (1)(10) - (2)(8) \\
            = -30
        \]

\item 
\begin{enumerate}
    \item We can begin by finding the area under the upper half of the ellipse. The integral is given by:
	    
	    \[
	        \int_{-a}^a \frac{b}{a}\sqrt{a^2-x^2}dx\\
	        =\frac{b}{a}\int_{-a}^a\sqrt{a^2-x^2}dx
	    \]
	    
	    We can use the trigonometric substitution $x = a\sin{\theta}, dx = a\cos{\theta}d\theta$:
	    
	    \begin{align*}
	        \frac{b}{a}\int_{-\frac{\pi}{2}}^\frac{\pi}{2}a\cos{\theta}a\cos{\theta}d\theta \\
	        = ab\int_{-\frac{\pi}{2}}^\frac{\pi}{2}\cos^2{\theta}d\theta\\
	        = ab\int_{-\frac{\pi}{2}}^\frac{\pi}{2}\frac{1}{2}(1+\cos{(2\theta)})d\theta\\
	        = \frac{ab}{2}\left[\frac{\pi}{2}-(-\frac{\pi}{2})\right]\\
	        = \frac{\pi ab}{2}
	    \end{align*}
	    
	    Using symmetry, the total area of the ellipse is double the above result, i.e., $\pi ab$.

        \item The trigonometric substitution solution is a bit more involved. First, we let $x = 3\sec{\theta}$:
	    
	    \begin{align*}
	        \int\frac{dx}{x^2-9} \\
	        = \int\frac{3\sec\theta\tan\theta}{9\tan^2\theta}d\theta \\
	        = \frac{1}{3}\int \csc\theta d\theta \\
	        = \frac{1}{3}\left(-\ln|\csc\theta + \cot\theta|\right) + C
	    \end{align*}
	    
	    We can draw a triangle representing the trigonometric substitution, giving us $\csc\theta = \frac{x}{\sqrt{x^2 - 9}}$ and $\cot\theta = \frac{3}{\sqrt{x^2 - 9}}$. Substituting this into the above expression and noting that $x > 3$ gives:
	    
	    \begin{align*}
	        \frac{1}{3}\left(-\ln\left|\frac{x+3}{\sqrt{x^2 - 9}}\right|\right) + C\\
	        = -\frac{1}{3}\ln\sqrt{\frac{x+3}{x-3}} + C\\
	        = \frac{1}{6}\ln\left(\frac{x-3}{x+3}\right) + C
	    \end{align*}

        \item A possible diagram that can represent this situation is given by a circle $x^2 + (y - a)^2 = 1$ for some constant $a$ that is tangent to two points on the cone $y = |x|$. The problem then becomes finding the area of the region between the circle and the cone. First, we must determine $a$ so that we can properly characterize the circle. Its equation is given by $y = a - \sqrt{1 - x^2}$. Since the circle is tangent to the cone, its slope must be given by: 
	    
	    \begin{align*}
	        \frac{dy}{dx} = -\frac{1}{2}\left(1 - x^2\right)^{-\frac{1}{2}}(-2x) \\
	        = \frac{x}{\sqrt{1 - x^2}}
	        = \pm 1
	    \end{align*}
	    
	    The points of tangency are also equal to the points of intersection. According to the above finding, the $x$ coordinate of of intersection is given by $x = \frac{1}{\sqrt{2}}$. Substituting this expression to the circle and cone equations gives:
	    
	    \begin{align*}
	        a - \sqrt{1 - \frac{1}{2}} = \left|\frac{1}{\sqrt{2}}\right| \implies a = \sqrt{2}
	    \end{align*}
	    
	    The area can then be given by the expression where we make use of symmetry about the y-axis:
	    
	    \begin{align*}
	        2\int_0^\frac{1}{\sqrt{2}}\left(\sqrt{2} - \sqrt{1 - x^2} - x\right)dx
	    \end{align*}
	    
	    The only somewhat tricky integral is given by the middle term of the integrand. Using a trigonometric substitution of $x = \sin\theta$, we should get an antiderivative of $\frac{1}{2}\sin^{-1}x + x\sqrt{1-x^2} + C$. The integral above then evaluates to:
	    
	    \begin{align*}
	        2\left[\sqrt{2}x - \frac{1}{2}\left(\sin^{-1}x + x\sqrt{1-x^2}\right) - \frac{x^2}{2}\right]_0^\frac{1}{\sqrt{2}} \\ 
	        = 1 - \sin^{-1}\frac{1}{\sqrt{2}} \\ 
	        = 1 - \frac{\pi}{4}
	    \end{align*}
\end{enumerate}


\end{enumerate}

	
	\end{solutions}
	\begin{instructions}
		\subsection*{Learning Objectives}
	Students need to be able to\ldots
	\begin{itemize}
		\item Practise integration techniques: integration by parts, trigonometric integrals, trigonometric substitution
	\end{itemize}

\subsection*{Notes}
	\begin{enumerate}
			\item This should be a relatively straightforward application of integration by parts. Part B may be tricky for students since there is only one function in the integrand.

            \item This question may trip students up since it requires them to think about how they can use a table of values to apply the integration by parts formula. Try to encourage students to write out the formula and then see if they can use the table of values to aid in evaluating the integral.

            \item All of these parts are meant to have students practise trigonometric substitution. The first two parts should be relatively straightforward, but the last one is very involved. Note that it is possible to solve it with geometry, but encourage students to try solving it using an integral for the practice.

            \item They are not likely to get to this question. This is meant to be an interesting application of integration by parts to show that the Gamma function satisfies an important property of the factorial function, and can extend it to other values. The actual integration by parts step itself isn't difficult, but there's a bit of a complication because they don't know integration by parts for infinite integrals yet, so things need to be phrased in terms of limits. 
	\end{enumerate}

	\end{instructions}

\end{document}
