		\begin{enumerate}
			\item 
            \begin{enumerate} 
            \item Since $f'(x) = -f(x)$ and $f(0) = 1$, the derivative will oscillate between $-1$ and $1$. The Taylor series is:

            \[\sum_{n=0}^\infty \frac{(-1)^nx^n}{n!}\]

            \item Simply plugging in $x^2$ and $\sqrt{x}$ into the series above gives:

            \[f(x^2)=\sum_{n=0}^\infty \frac{(-1)^nx^{2n}}{n!}, f(\sqrt{x}) = \sum_{n=0}^\infty \frac{(-1)^n(\sqrt{x})^n}{n!}\]

            On the other hand, we can take the derivative turn by turn to find:

            \[\frac{d}{dx}f(x) = \frac{d}{dx} \sum_{k
        n=0}^\infty \frac{(-1)^nx^n}{n!} = \frac{d}{dx}(1)+\sum_{n=1}^\infty \frac{n(-1)^nx^{n-1}} {n!}\]
            \[= \sum_{n=1}^\infty \frac{(-1)^nx^{n-1}}{(n-1)!}\]

            By shifting the indices to start at $n=0$, we find:

            \[=\sum_{n=0}^\infty \frac{(-1)^{n+1}x^{n}}{(n)!} = -f(x)\]

            Which is one way of seeing that $f'(x) = -f(x)$. 

            \item The series which are power series are those which only have integer powers, i.e. come about as infinite sums of polynomials. $f(\sqrt{x})$ is not a power series, since it has fractional powers in its expansion. 
            \end{enumerate}

            \item 
            \begin{enumerate}
                \item The term within the Taylor series representation of $e^x$ is given by $\displaystyle \frac{x^n}{n!}$, so we can apply the ratio test:
                \[\lim_{n\to \infty}\frac{\frac{\left|x^{n+1}\right|}{(n+1)!}}{\frac{|x|^n}{n!}} = \lim_{n\to\infty}\frac{|x|}{n+1} = 0\] 
                
                Regardless of the value for $x$, the ratio is always smaller than 1. Therefore, the Taylor series converges for all $x$.

                \item The term within the Taylor series representation of $\ln(1+x)$ is given by $\frac{(-1)^{n-1}x^n}{n}$, so applying the ratio test gives:
                \[\lim_{n\to\infty}\frac{\frac{|x^{n+1}|}{n+1}}{\frac{|x|^n}{n}} = \lim_{n\to\infty}\frac{1}{1+\frac{1}{n}}|x| = |x|\]
                
                Therefore, we require $|x| < 1$ for this series to converge.
            \end{enumerate}
            
            \item 
            \begin{enumerate}
                \item We know that $a_0 = 1, a_1 = -1, ..., a_n = (-1)^n$.
                
                
                \item To square the series representation, let's say that $g(x) = (f(x))^2=\sum_{n=0}^\infty b_n \frac{x^n}{n!}$. Writing out the first few terms of the expansion:

                \[(a_0 + a_1x + \frac{a_2}{2}x^2 +...)(a_0 + a_1x + \frac{a_2}{2}x^2 +...) = a_0^2 + 2a_0a_1x + (a_2a_0 +a_1^2)x^2 + ...\]
                We get this by fixing the degree of the term we care about and tracking the product terms with that degree. 
                In general, the $n$th degree term will look like:

                \[\frac{b_n}{n!} = \sum_{k=0}^n \frac{a_{n-k}a_k}{(n-k)!k!}\]
                \[ = \sum_{k=0}^n \frac{(-1)^n}{(n-k)!k!}\]

                
                More specifically:

                \begin{enumerate}
                    \item $b_0 = 1$
                    \item $b_1 = a_0a_1 + a_1 a_0 = -2$
                    \item $b_2/2! = a_2/2!a_0 + a_1^2 + a_0a_2/2! = a_0a_2 + a_1^2 = 2$
                    \item Finally we can find $b_3/3! = a_0\dfrac{a_3}{3!} + a_1 \dfrac{a_2}{2!} + \dfrac{a_2}{2!}a_1 + \dfrac{a_3}{3!}a_0 = -1-\frac{1}{3} = -\dfrac{4}{3}$

                    So the fourth order Taylor polynomial would be

                    \[1 -2x +2x^2 -\dfrac{4}{3}x^3 = 1 -2x +4\frac{x^2}{2!} -8\dfrac{x^3}{3!}\]

                    Where we rewrite the Taylor polynomial in a way that looks similar to the Taylor series we'll get in the next step. 
                    
                    
                \end{enumerate}

                

            

                \item We know that $g(x) = (e^{-x})^2 = e^{-2x}$. Taking the derivative $n$ times gives $g^{(n)}(0) = (-2)^n$. So the Taylor series at $x=0$ would be:

                \[g(x) = \sum_{n=0}^\infty (-2)^n\frac{x^n}{n!}\]

                \item They do! They should match because the function $f(x) = e^{-x}$ has a Taylor series which converges everywhere, which is shown in the next step!
            \end{enumerate}

            \item 
            \begin{enumerate}
                \item The first derivatives of $f(x) = e^{ix}$ are $f'(x) = ie^{ix}, f''(x) = -e^{ix}, f'''(x) = -ie^{ix}, f^{(4)}(x) = e^{ix}$, and so on. Therefore, the Taylor Series centred around $x=0$ would be $1 + ix -\frac{1}{2!}x^2 - \frac{i}{3!}x^3 + \frac{1}{4!}x^4 + \dots = \displaystyle \sum_{n=0}^\infty \frac{(ix)^n}{n!}$.

                \item Using the Taylor Series for $\cos{x}$ and $\sin{x}$, we get $\cos{x} + i\sin{x} = \sum_{n=0}^\infty \frac{(-1)^n}{(2n+1)!} x^{2n+1} + i\sum_{n=0}^\infty \frac{(-1)^n}{(2n)!}x^{2n} = 1 - \frac{x^2}{2!} + \frac{x^4}{4!} - \dots + i\left( x - \frac{x^3}{3!} + \dots\right)$

                \item We can factor out $i$ in the term by term expression in Part A to yield the same expression in Part B.
            \end{enumerate}            
		\end{enumerate}
