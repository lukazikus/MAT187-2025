
	
\begin{enumerate}
\item 
\begin{enumerate}
    \item Evaluate $\int \frac{1}{x^2 - 9}\,\mathrm dx$ for $x > 3$ using partial fractions.	
	The partial fractions solution is relatively straightforward where the integrand can be written as $\frac{1}{x^2 - 9} = \frac{A}{x-3} + \frac{B}{x+3}$ for some constants $A$ and $B$. The constants are given by $A = \frac{1}{6}$ and $B = -\frac{1}{6}$. The integral then becomes:
	    
    \[
        \int\frac{\mathrm dx}{x^2-9} \\
        = \int\frac{1/6}{x-3} - \frac{1/6}{x+3}\mathrm dx \\
        = \frac{1}{6}\ln|x-3| - \frac{1}{6}\ln|x+3| + C
    \]
    
    Since $x > 3$, we can drop the absolute value signs, and after some simplification, the expression simplifies to $\frac{1}{6}\ln\left(\frac{x-3}{x+3}\right) + C$.

    \item We can use the trigonometric substitution $x = 3\tan{\theta} \implies \mathrm dx = 3\sec^2\theta \mathrm d\theta$.

    Then we can write
    \[
    \int \frac{3\sec^2\theta\mathrm d\theta}{9\tan^2\theta + 9}\\
    =\int\frac{\sec^2\theta\mathrm d\theta}{3\sec^2\theta} = \frac{1}{3}\int\mathrm d\theta = \frac{1}{3}\theta+C = \frac{1}{3}\tan^{-1}\left(\frac{x}{3}\right)+C
    \]

    \item We can write the integrand as: 
    \[
    \frac{1}{(x+3i)(x-3i)} = \frac{i/6}{x+3i} - \frac{i/6}{x-3i}
    \]

    Then we can write this indefinite integral as:
    \[
    \frac{i}{6}\ln(x+3i) - \frac{i}{6}\ln(x-3i) + C = \frac{i}{6}\ln\left(\frac{x+3i}{x-3i}\right) + C
    \]

    Using the provided formula, we can rewrite the natural logarithm expression as:

    \[
    \ln1 + 2i\tan^{-1}\left(\frac{3}{x}\right) = 2i\tan^{-1}\left(\frac{3}{x}\right) = 2i\left[\frac{\pi}{2} - \tan^{-1}\left(\frac{x}{3}\right)\right]
    \]

    where the last inequality can be shown by drawing a triangle. Then the indefinite integral can be written as:
    \[
    \int\frac{\mathrm dx}{9+x^2} = \frac{1}{3}\tan^{-1}\left(\frac{x}{3}\right) - \frac{\pi}{6} + C = \frac{1}{3}\tan^{-1}\left(\frac{x}{3}\right) + D
    \]

    where we let another constant $D = - \frac{\pi}{6} + C$. This expression is equivalent to the one from Part B.
\end{enumerate}
\item 
\begin{enumerate}
    \item The work is given by:
    \[
    \displaystyle\int_{10}^1 -\frac{\left(8.99\times10^9\right)\left(1.6\times10^{-19}\right)^2}{r^2}dr = 2.30\times10^{-28}
    \]
    \item 
    No, this would require infinite energy. For any $D>0$
    $$\int_D^0 -k\frac{q_1 q_2}{r^2} \, dr = \underset{t\rightarrow 0}{\lim} \Big( k\frac{q_1 q_2}{t} -  k\frac{q_1 q_2}{D}\Big) = \infty$$
    \item Yes, this is given by the following improper integral which has a finite value:
    $$\int_\infty^1 -k\frac{q_1 q_2}{r^2} \, dr = \underset{t\rightarrow \infty}{\lim} \Big( k\frac{q_1 q_2}{1} -  k\frac{q_1 q_2}{t}\Big) = kq_1q_2$$
\end{enumerate}
\item 
\begin{enumerate}
    \item Open up Desmos!
    \item We would like to apply integration by parts. Recall the integration by parts formula:

    \[
    \int_0^\infty u\mathrm dv = uv|_0^\infty \int_0^\infty \mathrm du
    \]

    To the integral:
    \[
        \gamma(n) = \int_0^\infty e^{-t}t^{n-1}\,\mathrm dt.
    \]    
    Let $dv = e^{-t}\mathrm dt$ and $u = t^{n-1}$. Then $\mathrm d u = (n-1)t^{n-2} \mathrm d t$ and $v = -e^{-t}$. So:

    
    \[
        \int_0^\infty e^{-t}t^{n-1}\,\mathrm dt = -e^{-t}t^{n-1}|_0^\infty -\int_0^\infty -e^{-t}(n-1)t^{n-2}dt
    \]
    
    Since $-e^{-t}t^{n-1}|_0^\infty = 0$, we can see:
    \[
        =(n-1)\int_0^\infty e^{-t}t^{n-2}\mathrm dt =(n-1) \Gamma(n-1) 
    \]

    So:
    \[
        \Gamma(n) = (n-1)\Gamma(n-1) 
    \]

    \item We know that $-e^{-t}$ is an antiderivative for $e^{-t}$. So:

    \[
        \Gamma(1) = \int_0^\infty e^{-t} \mathrm dt
    \]


    
    \item In this case, $ \Gamma(4) = 3\Gamma(3) = 3\cdot 2 \Gamma(2) = 3\cdot 2 \cdot 1 \Gamma(1) = 3!$. 

    \item For general $n$, $\Gamma(n) = (n-1)!$.

    \item The recursive formula $\Gamma(x) = (x-1)\Gamma(x-1)$ will work for general $x > 1$, since we never used the fact that $n$ was an integer during integration by parts, only that $n>1$. It's only important for $n$ to be an integer so that we can relate $\gamma(n)$ to $\gamma(1)$. 

    
\end{enumerate}
\item 
    \begin{enumerate}
        \item 
        \begin{enumerate}
            \item For $x<1$, $g_x$ has a vertical asymptote. $g_x$ has a horizontal asymptote regardless of what $x$ is, and it is always nonnegative over the region $(0,\infty)$ that we care about for this question.
            \item  Since $x_1 \geq x_2\geq 1$, it follows $t^{x_1} \geq t^{x_2}$. Because all other factors are positive for $t>0$, $e^{-t}t^{x_1-1} \geq e^{-t}t^{x_2-1}$.
        \end{enumerate}
        \item For this part we assume that $x\geq 1$. Using the previous part, we can see that $g_n\geq g_x$ provided $x\leq n$. Since $g_x$ is positive by the comparison test, we can see that the following integral converges:

        \[
            \int_1^\infty g_x(t) \mathrm dt \leq \int_1^\infty g_n(t) <\infty 
        \]

        On the other hand, if $0< t \leq 1$, then $t^{x-1} \leq 1$, and $t^{x-1}e^{-t} \leq e^{-t}$. So we can apply the comparison test:

        \[
            \int_0^1 e^{-t}t^{x-1} \leq \int_0^1 e^{-t} = e^{-1} - 1
        \]
        So the integral converges. Since both these integrals converge, we know that the full integral converges, provided $x\geq 1$: 

        \[
           \int_0^\infty e^{-t} t^{x-1} \mathrm dt <\infty  
        \]
        
        \item Now we suppose that $0< x \leq 1$. 

        \begin{enumerate}
            \item Suppose that $0\leq t \le q1$. Then $e^{-t}\leq 1$, so the integrand $e^{-t}t^{x-1} \leq t^{x-1}$. So:

            \[
                \int_0^1 e^{-t}t^{x-1} \mathrm dt \leq \int_0^1 t^{x-1} \mathrm dt 
            \]

            Since $x>0$, $x-1>-1$. By the p-test, the integral on the right converges!

            \item Suppose that $1 \leq t \leq \infty$. Then $t^{x-1} \leq 1$. So $e^{-t}t^{x-1} \leq e^{-t}$. By the comparison test:

            \[
            \int_1^\infty e^{-t} t^{x-1} \leq \int_1^\infty e^{-t} = -e^{-t}|_0^\infty = 1
            \]
        \end{enumerate}
        Since both of these integrals converge, the full integral $\int_0^\infty e^{-t}t^{x-1}\mathrm dt$ converges! 
        
    \end{enumerate}
\end{enumerate}