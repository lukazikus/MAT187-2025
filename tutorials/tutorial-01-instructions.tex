		\subsection*{Learning Objectives} Students need to be able to\ldots
		\begin{itemize}
			\item Find a polynomial given a set of points that it goes through using Lagrange interpolation.
			\item Construct a Taylor polynomial and approximate its error by using Taylor's Remainder Theorem.
		\end{itemize}

		\subsection*{Context} 


		\subsection*{What to Do} This is the first tutorial of the term, and
		it is your chance to win the students over! This is a groupwork tutorial,
		but students may not be used to working in groups.

		\begin{itemize}
			\item Arranged for group work. Reorganize the desks and chairs
				(if possible) to facilitate groups of 3 or 4. Ask
				students to form groups of 3 or 4 with other students
				nearby. Don't allow larger groups.

			\item Begin the tutorial by introducing yourself (your name,
				your program of study, and your year). You might
				also want to give them some more personal information,
				such as where you are from or when you first started liking math.

			\item Introduce the structure and purpose of tutorials: students
				will be working to (1) better understand concepts
				from lecture, (2) practice tackling concepts that
				have not been explained in lecture, and (3) effectively
				communicate. They can expect to spend most of the
				tutorial working in small groups.

			\item Emphasize the importance of working with others when
				learning mathematics---they should be working with
				others in this tutorial \emph{and} outside of
				class.
		\end{itemize}

		This introduction should take no more than 5 minutes.

		Next, introduce the learning objectives for the day's tutorial. Explain the goal of this tutorial. Their worksheet has the ``formal'' objectives stated and these instructions have the ``hidden'' objectives. Feel free to share with them the hidden objectives as well.

		Ask the students to get together in groups and
		start working on the problem list. Circulate around the room during
		this time and ask groups what they're thinking. They will be tempted
		to move quickly through the list without thoroughly checking their
		new answers---encourage them to think deeply.

		There are too many problems to finish in 50 minutes and \emph{you should not be going
		over the solution to every problem}. Solutions will be posted for the students. The goal
		of tutorial is for students to spend time \emph{doing} mathematics with an expert around
		to help them if they get stuck. Don't feel any time pressure, even if you only get through 1.5
		questions, that's okay!

		During the last 6 minutes of class, pick one problem (perhaps a few parts of one problem)
		that most groups have at least started, and do this problem as a wrapup. Seeing an expert do the
		problem is the student's reward for working so hard.

		Notes:
		\begin{enumerate}

                \item 
                \item 
			\item Most groups will not have time to get to the last problem. That's okay! The moral is that you can prove some interesting identities using Taylor polynomials.

			Make sure the class knows that it's okay that they didn't get to this problem, but that it is good practice for them to do it at home.
		\end{enumerate}
