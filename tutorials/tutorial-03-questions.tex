		\begin{objectives}
			In this tutorial you practice manipulating Taylor series
		\end{objectives}

\subsection*{Problems}




\begin{enumerate}
    % Find a Taylor series. differentiate it. square it. integrate it.
    \item Consider the function $e^{\cos{x}}$. 
    
    \begin{enumerate}
        \item Write its Taylor series centred around $x=0$.
        \item Write an expression for the square, derivative, and indefinite integral of the series you found in Part A.
        \item Independent from the previous parts, write the Maclaurin series for the square, derivative, and indefinite integral of the above function.
        \item Compare your answers for Parts B and C. Do they match?
    \end{enumerate}

	\item In this questions, we will prove Euler's formula: $e^{ix} = \cos{x} + i\sin{x}$.
    \begin{enumerate}
        \item Write the Taylor series for $e^{ix}$ centred around $x=0$.
        \item Write a single series expression for $\cos{x} + i\sin{x}$ using the Taylor series for $\cos{x}$ and $\sin{x}$ centred around $x=0$.
        \item Manipulate your expression from Part A to match that of Part B.
    \end{enumerate}

    \item The angle $\Theta$ in radians of a robotic arm measured from the horizontal axis is known:

    \[\Theta(t) = 1 + \omega t - t^2 + R_2(t) \]

    Where $R_2(t)$ is an unknown error term with $|R_2(t)| \leq \dfrac{1}{10}t^3$, and $\omega$ is the angular velocity of the arm.  
    \begin{enumerate}

        \item Write down and simplify the expression for $\Theta^2$. 
        
        \item 

        
        \item If the length of the arm is $10$ metres, the horizontal position of the end is $x(t) = 10 \cos(\Theta(t))$. Without taking derivatives, find the second order Taylor polynomial for $x(t)$ based at zero. 
        
        \item  

        
    \end{enumerate}
		
\end{enumerate}
