		\begin{objectives}
			In this tutorial you practice setting up and manipulating Taylor series.
		\end{objectives}

\subsection*{Problems}

\begin{enumerate}
    % Find a Taylor series. differentiate it. square it. integrate it.
    \item Consider the function $f(x)=e^{-x}$. 
    
    \begin{enumerate}
        \item Write the Taylor series for $f$ centred at $x=0$.
        \item Find a series representation for the following functions by term-by-term manipulation of the Taylor series for $f$:
        \[
            g_1(x)=f(x^2)\qquad g_2(x)=f(\sqrt{x})\qquad g_3(x)=\frac{\mathrm d}{\mathrm d x}f(x)
        \]
        \item Which series from the previous part are \emph{power series}?

    \end{enumerate}

    \item 
    \begin{enumerate}
        \item Using the ratio test on the Taylor series  representation for $e^x$ centred around $x=0$, determine which values of $x$ that the Taylor series converges.
        \item Using the ratio test on the Taylo series representation for $\ln{(1+x)}$ centred around $x=0$, determine which values of $x$ that the Taylor series converges.
    \end{enumerate}

    \item Again, consider $f(x)=e^{-x}$ and its Taylor series representation at $x=0$,
    $\displaystyle f(x)= \sum_{n=0}^\infty a_n \frac{x^n}{n!}$
    \begin{enumerate}
        \item Write out the first five terms of the sequence $a_n$.
        \item Consider $g(x)=(f(x))^2$.  We will find the first few terms of the series representation for $g(x)$ by squaring the series representation for $f(x)$. 
        \begin{enumerate}
            \item What is the constant term in the series for $g$?
            \item What is the coefficient of the linear term (i.e., coefficient of $x$) in the series for $g$?
            \item What is the coefficient of the quadratic term in the series for $g$?
            \item Find the the first four terms of the series for $g$ (i.e. up to $x^3$ term).
        \end{enumerate}
        \item Find a series representation for $g(x)$ by directly computing the Taylor series for $g$ centered at $x=0$.

        \emph{Hint: it will be easier if you simplify your formula for $g$ before taking derivatives.}
        \item Do your series representations from the previous parts match? Do you expect them to always match?
    \end{enumerate}

	\item In this question, we will prove Euler's formula: $e^{ix} = \cos{x} + i\sin{x}$.
    \begin{enumerate}
        \item Write the Taylor series for $e^{ix}$ centred around $x=0$.
        \item Write a single series expression for $\cos{x} + i\sin{x}$ using the Taylor series for $\cos{x}$ and $\sin{x}$ centred around $x=0$.
        \item Manipulate your expression from Part A to match that of Part B.
    \end{enumerate}

\begin{comment}

    e^-x prove that for fixed x, it converges 
Do another with ratio test, finite ratio; for a fixed x, will it converge?
Show x^n/n! works for |x| < 1 

Start with sum of x^n, then find partial sums, bound them, show if |x|<1, it converges
Then consider e^x = sum x^n/n!, do comparison if |x|<1
Then do ratio test with e^x to find which x does this work for
Ratio test for Taylor series for ln(1+x)

    
    \item The angle $\Theta$ in radians of a robotic arm measured from the horizontal axis is known:

    \[\Theta(t) = \omega t - t^2 + R_2(t) \]

    Where $R_2(t)$ is an unknown error term with $|R_2(t)| \leq \dfrac{1}{10}t^3$, and $\omega$ is the angular velocity of the arm. 
    \begin{enumerate}

        \item Write down and simplify the expression for $\Theta^2$. Collect the constant, linear, and quadratic terms together, and leave the rest of the terms at the end. 

        
        \item If the length of the arm is $10$ metres, the vertical position of the end is $y(t) = 10 \sin(\Theta(t))$. Using the estimate $\sin(x) = x + R(x)$, where $R(x) \leq \dfrac{x^2}{2}$, find the second order Taylor polynomial for $y(t)$.
        
        \item  The remainder for part $b)$ is $R(\Theta(t)) + R_2(t)$. Find a bound on this error term.  

        \item Find a value of $\omega$ to make sure the robot arm catches a ball within a certain time

        
    \end{enumerate}
		
\end{comment}
\end{enumerate}
