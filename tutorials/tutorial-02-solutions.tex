\begin{enumerate}
    \item
    
    \item \begin{enumerate}
        \item 
    Picture a line segment of length $2$. Half of that segment corresponds to the $k=0$ term. Half of the remaining segment corresponds to the $k=1$ term. It goes on and on like that. 

    \item From the interpretation in part a), we can see that the tail $\sum_{m=k+1}^\infty \frac{1}{2^m} \leq \frac{1}{2^k}$. 

    \item We need the tail/error to satisfy $\frac{1}{2^k}\leq 0.01$. in other words, $\log_2(100)\leq k$. With a calculator $\log_2(100) \approx 6.64$, so $k\geq 7$ is enough. 
   
    \end{enumerate}
    \item
    \begin{enumerate}
        \item The total energy at time of impact is $E \approx 10\cdot 9.81$. Since ball loses $9$\% of the energy at impact, the highest the ball will travel after impact is $10\cdot 0.91 = 9.1$. After each impact, the height is reduced by a further factor of $0.91$. The initial distance traveled is $10$, and each distance after the initial bounce is $2\cdot10\cdot(0.91)^k$, where $k$ is the number of previous bounces. The distance after $N$ bounces, and just before the $N+1$st bounce is:
        
        \[D_N = 10+\sum_{k=1}^N 20 \cdot(0.91)^k\]

        \item 
    \end{enumerate}
    
    
        
\end{enumerate}