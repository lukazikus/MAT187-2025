	\begin{objectives}
        In this tutorial you practice manipulating integrals.
	\end{objectives}

		\vspace{-.5em}
		\subsection*{Problems}
		\vspace{-.5em}


\begin{enumerate}
    \item 
    \begin{enumerate}
        \item Compute $\displaystyle\int\log_b x\,dx$ where $b$ is any positive number.
        \item You are given that $\displaystyle\int_1^2 xf'(2x)\,dx=8$ and a table of values of $f$. Find $\displaystyle\int_1^2 f(2x)\,dx$.
    	\begin{align*}
    	    \hfill\begin{tabular}{c||c|c|c|c|c|c|c}
            $x$&$0$&$1$&$2$&$3$&$4$&$5$&$6$\\
            \hline
            $f(x)$&$2$&$-4$&$10$&$6$&$-2$&$8$&$-12$
            \end{tabular}\hfill\null
    	\end{align*}
    \end{enumerate}

    \item
    \begin{enumerate}
        \item Find the integral $\displaystyle\int\frac{dx}{x^2-2x+10}$.
        \begin{minipage}{\linewidth-5cm}
        \item The upper half of an ellipse centered at the origin with axes $a$ and $b$ is described by $y = \frac{b}{a}\sqrt{a^2-x^2}$ (see figure). Find the area of the ellipse in terms of $a$ and $b$.
        \end{minipage}\hfill
        \begin{minipage}{4cm}
    	\flushright
    	\begin{tikzpicture}[scale=0.45,line width=1]
    		\draw [tolTeal] (0,0) ellipse (3 and 2);
    			\draw[->] (-4,0) -- (4.2,0) node [below] {$x$};
    		\draw[->] (0,-2.5) -- (0,3.2) node [left] {$y$};
    		\node at (3,0) [below right,tolTeal] {$a$};
    		\node at (0,2) [above left,tolTeal] {$b$};
    	\end{tikzpicture}
    	\end{minipage}
    \end{enumerate}

    \item 
    \begin{enumerate}
        \item Evaluate $\int \frac{1}{x^2 - 9}\,dx$ for $x > 3$ using trigonometric substitution.
        \begin{minipage}{\linewidth-5cm}
        \item 
        A ball with radius 1 is placed inside a cone that has a vertical slope of 1. Determine the cross sectional area of the region underneath the ball but within the cone (grey in the figure).
        
        \textit{Hint: Use the fact that the ball must be tangent to the cone.}
        \end{minipage}\hfill
        \begin{minipage}{4cm}
	    \flushright
		\begin{tikzpicture}[scale=0.75,line width=1]
		\fill[gray] (0,0) -- ({1/sqrt(2)},{1/sqrt(2)}) arc (-45:-135:1) --cycle;
		\draw [tolMagenta] (0,{sqrt(2)}) circle (1);
		\draw [tolTeal] (-2,2) -- (0,0) -- (2,2);
		\draw[->] (-2,0) -- (2,0) node [below] {$x$};
		\draw[->] (0,-0.5) -- (0,3) node [left] {$y$};
	\end{tikzpicture}
	\end{minipage}
    \end{enumerate}

    \item The Gamma function is defined by the formula 

    \[\Gamma(x) = \int_0^\infty e^{-t}t^{x-1}dt\]
    
    It's an extension of the factorial function that works for non-integer values. If you ever graph $x!$ in desmos, this is what's being graphed!
    \begin{enumerate}
    \item Use the comparison test to show that the integral converges when $x>0$.
    
    \item Show by integration by parts that:

    \[\Gamma(x) = x\Gamma(x-1)\]
    \item Show that $\Gamma(1) = 1$ and $\Gamma(n) = n!$.
    \end{enumerate}    
\end{enumerate}

















