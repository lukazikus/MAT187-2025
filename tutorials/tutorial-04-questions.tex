	\begin{objectives}
        In this tutorial you practice manipulating integrals.
	\end{objectives}

		\vspace{-.5em}
		\subsection*{Problems}
		\vspace{-.5em}


\begin{enumerate}
    \item This question is about integration by parts. For each integral, identify what the ``parts'' are (i.e., the $u$ and $v$ in the formula
    $\displaystyle \int u\mathrm dv = uv-\int v\mathrm d u$) and then evaluate them.
    \begin{enumerate}
        \item $\displaystyle \int x\sin x\,\mathrm d x$
        \item $\displaystyle\int\log_b x\,\mathrm d x$ where $b>0$.
    \end{enumerate}
    \item 
    For an unknown function $f$, you know $\displaystyle\int_1^2 xf'(2x)\,\mathrm d x=8$ and 
    	\begin{align*}
    	    \hfill\begin{tabular}{c||c|c|c|c|c|c|c}
            $x$&$0$&$1$&$2$&$3$&$4$&$5$&$6$\\
            \hline
            $f(x)$&$2$&$-4$&$10$&$6$&$-2$&$8$&$-12$
            \end{tabular}\hfill\null
    	\end{align*}

        Use this information to find $\displaystyle\int_1^2 f(2x)\,\mathrm d x$.

    \item \phantom{x}
    \begin{enumerate}

        %\item Find the integral $\displaystyle\int\frac{dx}{x^2-2x+10}$.
        \item \phantom{x}
        
        \begin{minipage}{0.5\textwidth}
        The upper half of an ellipse centered at the origin with axes $a$ and $b$ is described by $y = \frac{b}{a}\sqrt{a^2-x^2}$ (see figure). Use an integral to find the area of the ellipse in terms of $a$ and $b$.

        \emph{Hint: you may want to try trigonometric substitution.}
        \end{minipage}%
        \begin{minipage}{0.5\textwidth}
            \begin{center}
                \begin{tikzpicture}[scale=0.45,line width=1]
                    \draw [tolTeal] (0,0) ellipse (3 and 2);
                        \draw[->] (-4,0) -- (4.2,0) node [below] {$x$};
                    \draw[->] (0,-2.5) -- (0,3.2) node [left] {$y$};
                    \node at (3,0) [below right,tolTeal] {$a$};
                    \node at (0,2) [above left,tolTeal] {$b$};
                \end{tikzpicture}
            \end{center}
    	\end{minipage}

        \item Evaluate $\int \frac{1}{x^2 - 9}\,\mathrm d x$ for $x > 3$ using trigonometric substitution.
        
        \item \phantom{x}
        
        \begin{minipage}{0.5\textwidth}
        A ball with radius 1 is placed inside a cone that has a vertical slope of 1. Set up an integral to determine the cross sectional area of the region underneath the ball but within the cone (grey in the figure).
        Then, use trigonometric substitution to evaluate the integral.
        
        \textit{Hint: Use the fact that the ball must be tangent to the cone.}

        \end{minipage}%
        \begin{minipage}{0.5\textwidth}
            \begin{center}
                \begin{tikzpicture}[scale=0.75,line width=1]
                    \fill[gray] (0,0) -- ({1/sqrt(2)},{1/sqrt(2)}) arc (-45:-135:1) --cycle;
                    \draw [tolMagenta] (0,{sqrt(2)}) circle (1);
                    \draw [tolTeal] (-2,2) -- (0,0) -- (2,2);
                    \draw[->] (-2,0) -- (2,0) node [below] {$x$};
                    \draw[->] (0,-0.5) -- (0,3) node [left] {$y$};
                \end{tikzpicture}
            \end{center}
    	\end{minipage}
    \end{enumerate}


    \newpage
    \item The Gamma function is defined by the formula 

    \[
        \Gamma(x) = \int_0^\infty e^{-t}t^{x-1}\,\mathrm dt
    \]
    It's an extension of the factorial function that works for non-integer values. If you ever graph $x!$ in desmos, this is what's being graphed!

    In this question, we will study the related antiderivative
    \[
        \gamma_b(n) = \int_0^b e^{-t}t^{n-1}\,\mathrm dt.
    \]
    

    Where $n$ is an integer greater than one.
    \begin{enumerate}
        \item Graph $x!$ in Desmos.
    
        \item Use integration by parts to find a simplified expression for:

        \[
            \gamma(n) - (n-1)\gamma(n-1) 
        \]

        \emph{Hint: Apply one integration by parts to $\gamma(n)$}
        \item Using the expression in part b), show that

        \[\lim_{b\to \infty} \gamma(n) - (n-1) \gamma(n-1) =0\]

        Provided $n\geq 1$.
        
        \item Assuming $\lim_{b\to \infty} \gamma(1) = 1$, can you determine what $\lim_{b\to \infty} \gamma(4)$ is?

        
        \item With the same assumptions, can you find a formula for $\lim_{b\to \infty}\gamma(n)$?
        \item Does your recursive formula for $\lim_{b\to \infty}\gamma$ still hold
        when computing $\lim_{b\to\infty}\gamma(x)$ when $x\in \R$ and $x\geq 1$? Explain.
    \end{enumerate}    
\end{enumerate}

















