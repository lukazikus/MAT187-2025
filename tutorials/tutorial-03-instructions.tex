\subsection*{Learning Objectives} Students need to be able to\ldots
	\begin{itemize}
		\item Expand and manipulate Taylor series
	\end{itemize}

\subsection*{What to Do} 
Introduce the learning objectives for the day's tutorial.
	Continue as usual, walking around the room and asking questions while letting students work on the next problem and gathering them together for discussion when most groups have finished.
    
    Seven minutes before class ends, pick a suitable problem to do as a wrap-up. Most likely, \#3 or \#4 will be a good choice.

\subsection*{Notes}
		\begin{enumerate}
			\item This question should be straight forward and serve
				as a warm-up for the tutorial. Don't spend a lot
				of time on this question.
			\item This question should be a fairly straightforward question about applying the ratio test. Be on the lookout for common mistakes such as not writing the limit symbol when applying the ratio test.

            \item In this question we don't want them to find the general formula for the terms of the square of the Taylor series, but if they're curious you can explain how it works. The idea for generating the first four terms is essentially the same as the general situation, thinking in terms of contributions from different degrees. It can be helpful for them to think of just the situation with a finite degree polynomial, and try to generalize from there. 

            If you find it appropriate, you can mention that since we expect the square of the Taylor series and the Taylor series of $e^{-2x}$ to agree, this means the coefficients line up. This is a neat way of seeing that $\sum_{k=0}^n {n \choose k} = 2^n$, although rewritten and better presented to the students as  $\sum_{k=0}^n \frac{1}{k!(n-k)!} = 2^n/n!$.
            
			\item This question should be relatively straightforward if students are comfortable with plugging in $ix$ as the argument within the Taylor series. One tricky part may be in Part C where they must recognize that factoring out the factor of $i$ is necessary to see how the answers for Parts A and B are equivalent.
		\end{enumerate}
