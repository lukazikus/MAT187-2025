\documentclass[red]{tutorial}
\usepackage[no-math]{fontspec}
\usepackage{xpatch}
	\renewcommand{\ttdefault}{ul9}
	\xpatchcmd{\ttfamily}{\selectfont}{\fontencoding{T1}\selectfont}{}{}
	\DeclareTextCommand{\nobreakspace}{T1}{\leavevmode\nobreak\ }
\usepackage{polyglossia} % English please
	\setdefaultlanguage[variant=us]{english}
%\usepackage[charter,cal=cmcal]{mathdesign} %different font
%\usepackage{avant}
\usepackage{microtype} % Less badboxes

%\usepackage{enumitem}

\usepackage[charter,cal=cmcal]{mathdesign} %different font
%\usepackage{euler}
 
\usepackage{blindtext}
\usepackage{calc, ifthen, xparse, xspace}
\usepackage{makeidx}
\usepackage[hidelinks, urlcolor=blue]{hyperref}   % Internal hyperlinks
\usepackage{mathtools} % replaces amsmath
\usepackage{bbm} %lower case blackboard font
\usepackage{amsthm, bm}
\usepackage{thmtools} % be able to repeat a theorem
\usepackage{thm-restate}
\usepackage{graphicx}
\usepackage{multicol}
\usepackage{fnpct} % fancy footnote spacing
\usepackage{tikz}
\usetikzlibrary{shapes,calc,arrows}
\usepackage{pgfplots}
\pgfplotsset{compat=1.15}
\usepackage{calc}
\usepackage{tabularx}
\newcolumntype{Y}{>{\centering\arraybackslash}X}
\usetikzlibrary{arrows.meta}
\newcommand{\checkbox}{\tikz[baseline=-1mm,line width=1]{\draw (0,0) circle (0.15);}}
%\usepackage{tcolorbox}

\definecolor{tolBlue}{HTML}{0077BB}
\definecolor{tolCyan}{HTML}{33BBEE}
\definecolor{tolTeal}{HTML}{009988} 
\definecolor{tolOrange}{HTML}{EE7733} 
\definecolor{tolRed}{HTML}{CC3311} 
\definecolor{tolMagenta}{HTML}{EE3377} 
\definecolor{tolGrey}{HTML}{BBBBBB}

\usepackage{tcolorbox}
\definecolor{tolBlue}{HTML}{0077BB}
\definecolor{tolCyan}{HTML}{33BBEE}
\definecolor{tolTeal}{HTML}{009988} 
\definecolor{tolOrange}{HTML}{EE7733} 
\definecolor{tolRed}{HTML}{CC3311} 
\definecolor{tolMagenta}{HTML}{EE3377} 
\definecolor{tolGrey}{HTML}{BBBBBB}
\usepackage{enumitem}
%\pgfkeys{/pgf/fpu}

 
\newcommand{\xh}{{{\mathbf e}_1}}
\newcommand{\yh}{{{\mathbf e}_2}}
\newcommand{\zh}{{{\mathbf e}_3}}
\newcommand{\R}{\mathbb{R}}
\newcommand{\Z}{\mathbb{Z}}
\newcommand{\N}{\mathbb{N}}
\newcommand{\proj}{\mathrm{proj}}
\newcommand{\Proj}{\mathrm{proj}}
\newcommand{\Perp}{\mathrm{perp}}
\renewcommand{\span}{\mathrm{span}\,}
\newcommand{\Span}{\mathrm{span}\,}
\newcommand{\Img}{\mathrm{img}\,}
\newcommand{\Null}{\mathrm{null}\,}
\newcommand{\Range}{\mathrm{range}\,}
\newcommand{\rref}{\mathrm{rref}}
\newcommand{\rank}{\mathrm{rank}}
\newcommand{\Rank}{\mathrm{rank}}
\newcommand{\nnul}{\mathrm{nullity}}
\newcommand{\mat}[1]{\begin{bmatrix}#1\end{bmatrix}}
\newcommand{\chr}{\mathrm{char}}
\renewcommand{\d}{\mathrm{d}}


\theoremstyle{definition}
\newtheorem{example}{Example}[section]
\newtheorem{defn}{Definition}[section]

%\theoremstyle{theorem}
\newtheorem{thm}{Theorem}[section]

\pgfkeys{/tutorial,
	name={Tutorial 6},
	author={},
	course={MAT 187},
	date={},
	term={},
	title={Parametric Equations}
	}

\begin{document}
	\begin{tutorial}
		\begin{objectives}
    In this tutorial you will be exploring parametric equations.
\end{objectives}

\begin{enumerate}
    \item Consider a particle following a \emph{Lissajous curve} trajectory that is given by the parametric equations
    \[
    \begin{cases}
        x=\sin{4t}\\
        y=\sin{3t}
    \end{cases}\qquad 0 \leq t \leq 2\pi
    \]

    \begin{enumerate}
        \item Consider the derivatives $\frac{dx}{dt}$ and $\frac{dy}{dt}$. 
        When the particle is moving left, what must be true about $\frac{dx}{dt}$ and $\frac{dy}{dt}$?
        How about when the particle is moving right? Up? Down?

        \item Find a $t$ when the instantaneous velocity of the particle is moving diagonal, with a trajectory towards the upper left. Justify your answer.

        \item Find \textbf{all times} $t$ during which the particle is moving
        to the upper left.
        
    \end{enumerate}

    \item Consider the following curves given by parametric equations.

    \begin{tcolorbox}[sharp corners=all,colframe=tolGrey,colback=white]
    \begin{multicols}{2}
    \begin{enumerate}[label={(\roman{enumii})},nosep,itemsep=1mm]
        \item $x = \cos{2t}, y = \sin{2t}, 0 \leq t \leq \pi$
        \item $x = \cos{t}, y = \sin{t}, 0 \leq t \leq 2\pi$
        \item $x = \sin{t}, y = \cos{t}, 0 \leq t \leq 2\pi$
        \item $x = \cos{2t}, y = \cos{2t}, 0 \leq t \leq \pi$
        \item $x = \cos{t}, y = \sin{t}, 0 \leq t \leq \pi$
    \end{enumerate}
    \end{multicols}
    \end{tcolorbox}
    
    \begin{enumerate}
        \item Sketch the graph (by hand) of each set of parametric equations.
        Add arrows to your sketches to signify which direction each curve is traced out.

        \item Which parametric equations produce the same shapes when plotted in an $xy$-coordinate system? 
        
        Justify your answer by plotting in Desmos
        
        \url{https://www.desmos.com/calculator/24kii8joui}

        Note: \emph{make sure to appropriately adjust the domains of the parametric curves when you plot in Desmos.}
        
    \end{enumerate}

    % \item Consider constants $a,b,c,d$ such that $a^2+c^2 = b^2+d^2 = R^2$ and $ab+cd=0$. Show that the parametric equations below describe a circle of radius $R$.
    % \[
    %     \begin{cases}
    %     x = a\cos{t} + b\sin{t}\\
    %     y = c\cos{t} + d\sin{t}
    %     \end{cases}
    % \]

    \item A colony of ants is marching according to the trajectory $x(t) = \cos{t}, y(t)=\sin{3t}$. 
    \begin{enumerate}
    \item Graph $t$ vs.{} $x$ and $t$ vs.{} $y$.
    \item  Use your plots from part (a) to sketch the
    graph of $(x(t),y(t))$ for $t\in \R$.

    Verify your sketch using Desmos.

    \item Is there a tangent line to the graph at $(1/2,0)$ Why or why not?

    \item If we restrict the domain of $(x(t),y(t))$ to $t\in[0,\pi]$ is
    there a tangent line to the graph at $(1/2,0)$? If so, find it.
    
    \item If we restrict the domain of $(x(t),y(t))$ to $t\in[\pi,2\pi]$ is
    there a tangent line to the graph at $(1/2,0)$? If so, find it.

    \end{enumerate}
    
\end{enumerate}
	\end{tutorial}

	\begin{solutions}
		\begin{enumerate}
	\item 
    The question is asking for one time when $\frac{dx}{dt}<0$ and $\frac{dy}{dt}>0$.
    
        Note that this solution provides more information than what is asked of the students. They only need to find one moment in time.
    
        The particle is moving left when $\frac{dx}{dt} < 0$, and it is moving right when $\frac{dx}{dt} > 0$. Similarly, the particle is moving up when $\frac{dy}{dt} > 0$, and it is moving down when $\frac{dy}{dt} < 0$. 
        
        For moving left, we have:
\[
    \begin{aligned}
       4\cos{4t} < 0\\
            \implies \cos{4t} < 0\\
            \implies \frac{\pi}{2} < 4t < \frac{3\pi}{2}, \frac{5\pi}{2} < 4t < \frac{7\pi}{2}, \frac{9\pi}{2} < 4t < \frac{11\pi}{2}, \frac{13\pi}{2} < 4t < \frac{15\pi}{2}\\
            \implies \frac{\pi}{8} < t < \frac{3\pi}{8}, \frac{5\pi}{8} < t < \frac{7\pi}{8}, \frac{9\pi}{8} < t < \frac{11\pi}{8}, \frac{13\pi}{8} < t < \frac{15\pi}{8}
    \end{aligned}
\]
        
        For moving right, we can simply take the complementary times:
        \[
            0 < t < \frac{\pi}{8}, \frac{3\pi}{8} < t < \frac{5\pi}{8}, \frac{7\pi}{8} < t < \frac{9\pi}{8}, \frac{11\pi}{8} < t < \frac{13\pi}{8}, \frac{15\pi}{8} < t < 2\pi
        \]
        For moving down, we can write a very similar set of inequalities, except now $3t$ is the main argument instead of $4t$ and we exclude one of the inequalities to stay within the required bound for $t$:
        
        \[
        \begin{aligned}
            &3\cos{3t} < 0\\
            \implies &\cos{3t} < 0\\
            \implies &\frac{\pi}{2} < 3t < \frac{3\pi}{2}, \frac{5\pi}{2} < 3t < \frac{7\pi}{2}, \frac{9\pi}{2} < 3t < \frac{11\pi}{2}, \frac{13\pi}{2} < 3t < \frac{15\pi}{2}\\
            \implies &\frac{\pi}{6} < t < \frac{\pi}{2}, \frac{5\pi}{6} < t < \frac{7\pi}{6}, \frac{3\pi}{2} < t < \frac{11\pi}{6}
        \end{aligned}
        \]
        Like before, we determine when the particle is moving up by finding the complementary times:
        \[
            0 < t < \frac{\pi}{6}, \frac{\pi}{2} < t < \frac{5\pi}{6}, \frac{7\pi}{6} < t < \frac{3\pi}{2}, \frac{11\pi}{6} < t < 2\pi
        \]
        
        To determine a time $t$ where the particle moves left and up, we can take the intersection between the intervals found for moving left and moving up:
        
        \[
            \frac{\pi}{8} < t < \frac{\pi}{6}, \frac{5\pi}{8} < t < \frac{5\pi}{6}, \frac{7\pi}{6} < t < \frac{11\pi}{8}, \frac{11\pi}{6} < t < \frac{15\pi}{8}
        \]

    \item 
    \begin{enumerate}
        \item All plots are circles with radius 1 centered at the origin, except for set 4 which is just the line segment $y=x$ from $(0,0)$ to $(1,1)$ and set 5 which is a semi-circle with radius 1 defined for $y > 0$. The direction for the circles is counterclockwise except set 3 which is clockwise. For the line, the direction is down-left followed by up-right since $t=0$ corresponds to the coordinate $(1,1)$, $t=\frac{\pi}{2}$ corresponds to $(-1,-1)$, and $t=\pi$ corresponds to $(1,1)$.

        \item All sets except the 4th and 5th ones are equivalent to each other. Their shapes are circles with radius 1 centered at the origin.
    \end{enumerate}

    \item
    We first plot the functions $x=\cos{t}, y=\sin{3t}$. Tracing $0 < t < 2\pi$, we can plot the parametric equation on the $x-y$ plane.
	
	    \begin{figure}[!ht]
	    \centering
    
        \begin{tikzpicture}[line cap=round,line join=round,>=triangle 45,x=1cm,y=1cm]
            \begin{axis}[
                x=2cm,y=2cm,
                axis lines=middle,
                ymajorgrids=true,
                xmajorgrids=true,
                xmin=-0.22083994729243386,
                xmax=6.549381871767769,
                ymin=-2.164789358246741,
                ymax=2.0599704714017775,
            ]
            \clip(-0.22083994729243386,-2.164789358246741) rectangle (6.549381871767769,2.0599704714017775);
            \draw[line width=2pt,color=black,smooth,samples=100,domain=-0.22083994729243386:6.549381871767769] plot(\x,{cos(((\x))*180/pi)});
            \draw[line width=2pt,color=blue,smooth,samples=100,domain=-0.22083994729243386:6.549381871767769] plot(\x,{sin((3*(\x))*180/pi)});
            \begin{scriptsize}
                \draw[color=black] (6, 1.1) node {$x=\cos{t}$};
                \draw[color=blue] (6,0.5) node {$y=\sin{3t}$};
            \end{scriptsize}
            \end{axis}
        \end{tikzpicture}
    
        \end{figure}

        See \href{https://www.desmos.com/calculator/xdoroqqzdz}{here} for an interactive Desmos plot of the paths that the ants take. Feel free to play around with the parameter $t$!
        
        To find the coordinates of the horizontal and vertical tangents, we can set $\frac{dx}{dt}=-\sin{t}=0$ and $\frac{dy}{dt}=3\cos{3t}=0$, respectively. This corresponds to $t=0, \pi$ and $t=\frac{\pi}{6}, \frac{\pi}{2}, \frac{5\pi}{6}, \frac{7\pi}{6}, \frac{3\pi}{2}, \frac{11\pi}{6}$, respectively. If we plug these values of $t$ into the parametric equation, we find that the vertical tangents are located at (1,0) and (-1,0). The horizontal tangents are located at $\left(\frac{\sqrt{3}}{2},1\right), (0,-1), \left(-\frac{\sqrt{3}}{2},1\right), \left(-\frac{\sqrt{3}}{2},-1\right), (0,1), \left(\frac{\sqrt{3}}{2},-1\right)$.
        

\end{enumerate}
	
	\end{solutions}
	\begin{instructions}
		\subsection*{Learning Objectives}


\subsection*{Notes}
	\begin{enumerate}
    \item This should be a straightforward question. 
    \item Note: (iv) is a straight diagonal line. All others are paths on the circle, with different speeds and domains for the angles. Students will have to relate the speed (the coefficient on $t$) and the size of the domain to the graphical drawing.
    \item They may struggle knowing what the equation for the tangent line is or what the tangent line is conceptually. They need to write down a formula for the line. 
\end{enumerate}
	\end{instructions}

\end{document}
