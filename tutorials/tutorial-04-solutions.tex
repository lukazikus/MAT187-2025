\begin{enumerate}
\item 
\begin{enumerate}
    \item Choosing $u=x, dv=\sin{x}\mathrm dx$, the integral becomes 
    \[
    -x\cos{x} - \int-\cos{x}\mathrm dx
    =-x\cos{x} + \sin{x} + C
    \]

    \item Using integration by parts, we can let $u = \log_b x, \mathrm dv = \mathrm dx$, giving $\mathrm du = \frac{dx}{x\ln b}, v = x$. Then we get
		
		\[
			\int\log_b x \mathrm dx \\
			= x\log_b x - \int\frac{x\mathrm dx}{x\ln b} \\
			= x\log_b x - \frac{x}{\ln b} + C \\
			= \frac{1}{\ln b}(x\ln x - x) + C
		\]
\end{enumerate}
\item
Using integration by parts, we can let $u = f(2x), \mathrm dv = \mathrm dx$, giving $\mathrm du = 2f'(2x)\mathrm dx, v = x$. Applying the integration by parts formula gives:
        
        \[
            \int_1^2 f(2x)\mathrm dx \\
            = \left[xf(2x)\right]_1^2 - \int_1^2 2xf'(2x)\mathrm dx \\
            = (2)(-2) - (1)(10) - (2)(8) \\
            = -30
        \]

\item 
\begin{enumerate}
    \item We can begin by finding the area under the upper half of the ellipse. The integral is given by:
	    
	    \[
	        \int_{-a}^a \frac{b}{a}\sqrt{a^2-x^2}dx\\
	        =\frac{b}{a}\int_{-a}^a\sqrt{a^2-x^2}dx
	    \]
	    
	    We can use the trigonometric substitution $x = a\sin{\theta}, dx = a\cos{\theta}d\theta$:
	    
	    \begin{align*}
	        \frac{b}{a}\int_{-\frac{\pi}{2}}^\frac{\pi}{2}a\cos{\theta}a\cos{\theta}d\theta \\
	        = ab\int_{-\frac{\pi}{2}}^\frac{\pi}{2}\cos^2{\theta}d\theta\\
	        = ab\int_{-\frac{\pi}{2}}^\frac{\pi}{2}\frac{1}{2}(1+\cos{(2\theta)})d\theta\\
	        = \frac{ab}{2}\left[\frac{\pi}{2}-(-\frac{\pi}{2})\right]\\
	        = \frac{\pi ab}{2}
	    \end{align*}
	    
	    Using symmetry, the total area of the ellipse is double the above result, i.e., $\pi ab$.

        \item The trigonometric substitution solution is a bit more involved. First, we let $x = 3\sec{\theta}$:
	    
	    \begin{align*}
	        \int\frac{dx}{x^2-9} \\
	        = \int\frac{3\sec\theta\tan\theta}{9\tan^2\theta}d\theta \\
	        = \frac{1}{3}\int \csc\theta d\theta \\
	        = \frac{1}{3}\left(-\ln|\csc\theta + \cot\theta|\right) + C
	    \end{align*}
	    
	    We can draw a triangle representing the trigonometric substitution, giving us $\csc\theta = \frac{x}{\sqrt{x^2 - 9}}$ and $\cot\theta = \frac{3}{\sqrt{x^2 - 9}}$. Substituting this into the above expression and noting that $x > 3$ gives:
	    
	    \begin{align*}
	        \frac{1}{3}\left(-\ln\left|\frac{x+3}{\sqrt{x^2 - 9}}\right|\right) + C\\
	        = -\frac{1}{3}\ln\sqrt{\frac{x+3}{x-3}} + C\\
	        = \frac{1}{6}\ln\left(\frac{x-3}{x+3}\right) + C
	    \end{align*}

        \item A possible diagram that can represent this situation is given by a circle $x^2 + (y - a)^2 = 1$ for some constant $a$ that is tangent to two points on the cone $y = |x|$. The problem then becomes finding the area of the region between the circle and the cone. First, we must determine $a$ so that we can properly characterize the circle. Its equation is given by $y = a - \sqrt{1 - x^2}$. Since the circle is tangent to the cone, its slope must be given by: 
	    
	    \begin{align*}
	        \frac{dy}{dx} = -\frac{1}{2}\left(1 - x^2\right)^{-\frac{1}{2}}(-2x) \\
	        = \frac{x}{\sqrt{1 - x^2}}
	        = \pm 1
	    \end{align*}
	    
	    The points of tangency are also equal to the points of intersection. According to the above finding, the $x$ coordinate of of intersection is given by $x = \frac{1}{\sqrt{2}}$. Substituting this expression to the circle and cone equations gives:
	    
	    \begin{align*}
	        a - \sqrt{1 - \frac{1}{2}} = \left|\frac{1}{\sqrt{2}}\right| \implies a = \sqrt{2}
	    \end{align*}
	    
	    The area can then be given by the expression where we make use of symmetry about the y-axis:
	    
	    \begin{align*}
	        2\int_0^\frac{1}{\sqrt{2}}\left(\sqrt{2} - \sqrt{1 - x^2} - x\right)dx
	    \end{align*}
	    
	    The only somewhat tricky integral is given by the middle term of the integrand. Using a trigonometric substitution of $x = \sin\theta$, we should get an antiderivative of $\frac{1}{2}\sin^{-1}x + x\sqrt{1-x^2} + C$. The integral above then evaluates to:
	    
	    \begin{align*}
	        2\left[\sqrt{2}x - \frac{1}{2}\left(\sin^{-1}x + x\sqrt{1-x^2}\right) - \frac{x^2}{2}\right]_0^\frac{1}{\sqrt{2}} \\ 
	        = 1 - \sin^{-1}\frac{1}{\sqrt{2}} \\ 
	        = 1 - \frac{\pi}{4}
	    \end{align*}
\end{enumerate}

% \item 
% \begin{enumerate}
%     \item Open up Desmos!
%     \item We would like to apply integration by parts. Recall the integration by parts formula:

%     \[
%     \int_0^b u\mathrm dv = uv|_0^b - \int_0^b v\mathrm du
%     \]

%     To the integral:
%     \[
%         \gamma(n) = \int_0^b e^{-t}t^{n-1}\,\mathrm dt.
%     \]
    
%     Let $dv = e^{-t}\mathrm dt$ and $u = t^{n-1}$. Then $\mathrm d u = (n-1)t^{n-2} \mathrm d t$ and $v = -e^{-t}$. So:

    
%     \[
%         \int_0^b e^{-t}t^{n-1}\,\mathrm dt = -e^{-t}t^{n-1}|_0^b -\int_0^b -e^{-t}(n-1)t^{n-2}dt
%     \]
    
%     Since $-e^{-t}t^{n-1}|_0^b = -e^{-b}b^{n-1}$, we can see:
%     \[
%         =(n-1)\int_0^b e^{-t}t^{n-2}\mathrm dt =(n-1) \gamma(n-1)  -e^{-b}b^{n-1}
%     \]

%     So:
%     \[
%         \gamma(n) - (n-1)\gamma(n-1) = e^{-b}b^{n-1}
%     \]

%     \item Provided $n\geq 1$,

%     \[\lim_{b\to \infty} e^{-b}b^{n-1} = 0\]

    
%     \item In this case, $\lim_{b\to \infty} \gamma(4) = 3\lim_{b\to \infty}\gamma(3) = 3\cdot 2 \lim_{b\to \infty}\gamma(2) = 3\cdot 2 \cdot 1 \gamma(1) = 3!$. 

%     \item For general $n$, $\lim_{b\to \infty}\gamma(n) = (n-1)!$.

%     \item The recursive formula $\lim_{b\to \infty}\gamma(x) = (x-1)\lim_{b\to \infty}\gamma(x-1)$ will work for general $x \geq 1$, since we never used the fact that $n$ was an integer during integration by parts. It's only important for $n$ to be an integer so that we can relate $\gamma(n)$ to $\gamma(1)$. 

    
% \end{enumerate}



\end{enumerate}
