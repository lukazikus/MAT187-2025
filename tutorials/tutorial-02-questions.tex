		
\begin{objectives}
	In this tutorial you practice using and manipulating sequences and series.
\end{objectives}

	\vspace{-1em}
\subsection*{Problems}
\begin{enumerate}
	\item 
    
        \begin{enumerate}
			\item Consider $A=\displaystyle \sum_{i=2}^\infty 3^{i}$. Rewrite $A$ (using summation notation) as
			a sum starting from $i=0$.
            \item %Practice shifting indices to match terms
            Consider  $A=\displaystyle \sum_{i=2}^\infty 3^{i}$ and
             $
			 B=\displaystyle \sum_{i=0}^\infty 2^{i}$.

            Write down an expression for $A+B$ using a single summation.

			\emph{Hint:} Pay close attention to the indices!

            \item %Practice pulling terms from the start or end of a sum
            Consider the series $\displaystyle \sum_{i=1}^\infty \frac{i^3}{2+e^{i-1}}$. Rewrite the series so that it starts at $i=4$.
            
            %\item Suppose we have the following partial sums: $\displaystyle \displaystyle \sum_{i=0}^n \ln{i}$, $\displaystyle \sum_{i=0}^n \ln{i^2}$, $\displaystyle \sum_{j=1}^n \log_{10}{j}, \displaystyle \sum_{j=3}^n \ln{j}, \displaystyle \sum_{j=1}^n \ln{\frac{j}{3}}$. Using index shifting and properties of logarithms, write an expression for the sum of these partial sums that includes only one summation.
            
        \end{enumerate}% Some questions on summation notation
        
        \item %Geometric Series

        Let 
        
        \[S_n = \sum^n_{k=0} \left(\tfrac{1}{2}\right)^k\]

        \begin{enumerate}
            \item Interpret $S_n$ geometrically, in terms of length or area.

            \item Define $S_\infty=\lim_{n\to\infty}S_n$. The \emph{tail sum} of sum $S_n$
			is the number $S_\infty - S_n$. 

			Use your interpretation from part (a) to come up with a bound for the tail sum of $S_n$.

            \item How large must $n$ be so that $S_n$ is within $0.01$ of $S_\infty$?
		\end{enumerate}

		\item A \emph{geometric} series is one that takes the form $\displaystyle S_n=\sum_{i=0}^n \alpha^i$.
		\begin{enumerate}
			\item Compute the quantity $S_n-\alpha S_n$. Simplify your answer.
			\item Solve $S_n-\alpha S_n=\,??$ for $S_n$, where $??$ is filled in with your answer from the previous part. 
			\item Give a closed-form formula for $S_n$ (i.e., one that does not involve summation notation).
			\item Use your formula to compute $S_\infty =\lim_{n\to\infty} S_n$.
		\end{enumerate}

        \item A ball is dropped from a height of 10 metres. Each time the ball bounces, the maximum height of the bounce is $9$\% less than the maximum height of the previous bounce. 
        
        \begin{enumerate}
            \item Write down a series that represents the total distance the ball has travelled when it hits the ground for the $N$th time. 
            \item After how many bounces can we be certain the ball won't bounce higher than $10$cm?
            \item After how many bounces can we be certain the the total distance the ball moves (i.e., combining all future bounces) won't be larger than $10$cm?
            \item What is the total distance the ball will travel (i.e., combining all bounces)
			over its entire lifetime?
        \end{enumerate}

	% Some questions where tail sums are bounded
    % Give series with bound, then use error estimate, don't use x
    % Draw picture to show bound
\end{enumerate}