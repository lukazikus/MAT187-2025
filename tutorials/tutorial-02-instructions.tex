\subsection*{Learning Objectives}
	Students need to be able to\ldots
	\begin{itemize}
		\item Manipulate series by shifting indices.
        \item Work with geometric series and their applications.
	\end{itemize}

\subsection*{What to Do}
	Introduce the learning objectives for the day's tutorial.
	Continue as usual, walking around the room and asking questions while letting students work on the next problem and gathering them together for discussion when most groups have finished.
    
    Seven minutes before class ends, pick a suitable problem to do as a wrap-up. Most likely, \#3 or \#4 will be a good choice.
	
\subsection*{Notes}
	\begin{enumerate}
		\item Hopefully this problem won't be hard for them. The only tricky part may be understanding how to shift indices, but if you get them to write out a few terms, they should be able to see how they should modify the index within the summation.

		\item The goal for this question is to get across the value of geometric interpretations for creating inequalities, and using that to get a sense of the error of a finite approximation.

		\item This is a standard proof of the geometric series formula. If they struggle with manipulating the series in part 1, they will probably struggle here as well. 

		\item They may not get to this question, that's okay. There is a subtle point to be made while constructing the sum: before the first drop, the ball will only travel $10$m. After the first bounce, the ball will travel twice the maximum height it reaches. In addition, exactly what the bounds of the sum should be is a bit subtle. Since we care about the distance just after the $N$th bounce, we do not include the distance travelled during the $N$th bounce.

	\end{enumerate}