\begin{enumerate}
    \item 

        \begin{enumerate}
        \begin{enumerate}
            \item Linear, separable
            \item Nonlinear, separable
            \item Linear, separable
            \item Nonlinear, nonseparable
            \item Linear, nonseparable
            \item Linear, separable
            \item Nonlinear, separable
            \item Linear, nonseparable
            \item Nonlinear, separable
            \item Linear, nonseparable
            
        \end{enumerate}
        \end{enumerate} 
    \item 
    $y=1$ is semistable,
    $y=2$ is stable,
    $y=3$ is unstable
    
    A visual argument (``If we are just below 1, the derivative is positive, so it's going back to 1'') is nicer than an abstract argument that seems memorized (``If the phase plot crosses from top left to bottom right, it is stable'')

    \item 
    We first note that all ODEs are first-order and autonomous, meaning we can look at their roots to find their equilibrium points and match them with the given direction fields. 
    
    Direction field (A) shows one equilibrium solution at $y = 2$. Therefore, either $y' = y - 2$ or $y'= 2 - y$ are valid candidates. We also see that for $y > 2$, the solutions are decreasing (i.e., $y' < 0$) while for $y < 2$, the solutions are increasing (i.e., $y' > 0$). Therefore, $y' = 2 - y$ must be the matching ODE.
    
    A similar line of reasoning applies for direction field (B) where now we see that for $y > 2$, the solutions are increasing (i.e., $y' > 0$) while for $y < 2$, the solutions are decreasing (i.e., $y' < 0$). Therefore, $y' = y - 2$ must be the matching ODE.
    
    Matching direction fields (C) and (D) follows a similar line of reasoning as matching direction fields (A) and (B). The only difference is that the equilibrium point is at $y = -2$, meaning either $y' = 2 + y$ or $y' = -2-y$ are valid ODEs. For direction field (C), $y' < 0$ for $y > -2$ and $y' > 0$ for $y < -2$. Therefore, $y' = -2 - y$ is the matching ODE. This leaves direction field (D) to match $y' = 2 + y$.
    
    Direction fields (E) and (F) have equilibrium points at $y = 0$ and $y = 3$, meaning $y' = y(3-y)$ and $y' = y(y-3)$ are potential candidates. For direction field (E), $y' < 0$ for $y < 0, y > 3$ and $y' > 0$ for $0 < y < 3$. Therefore, $y' = y(3 - y)$ is the matching ODE. This leaves direction field (F) to match $y' = y(y-3)$.
\end{enumerate}