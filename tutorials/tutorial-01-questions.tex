\begin{objectives}
	In this tutorial, you practice creating and using polynomial approximations of functions.
\end{objectives}

%	\bigskip


\subsection*{Problems}

\begin{enumerate}
	\item  The inverse square root function, $f(x) = 1/\sqrt{x}$ is an important function
	      when simulating light and reflections in computer graphics.  However, computing square
	      roots can be expensive, so we will use much-simpler polynomial approximations.
	      \begin{enumerate}
		      \item Find the first and second degree Taylor approximations to $f$ centered at $4$.
		            Use each of these polynomials to approximate $f(2)$. Use a calculator
		            to quantify the error in each approximation.
		      \item Find the quadratic polynomial that interpolates between the points
		            $(3,f(3))$, $(4,f(4))$, and $(5,f(5))$.
		            Use this polynomial to approximate $f(2)$ and quantify its error.

		      \item Use Desmos to graphically answer the following:
		            In what region(s) is the interpolating polynomial better?
		            In what region(s) is the Taylor polynomial better?

		      \item Find the third and fourth degree Taylor approximations to $f$ centered at $4$.
		            How well do they approximate $f(0.5)$. What about $f(7.5)$?

		      \item Write down the remainder formula for the $n$th order
		            Taylor approximation of $f$.
		            What does this formula say about the error of your Taylor approximations near $0$?
		            What about near $8$?
	      \end{enumerate}
          \clearpage

	\item The step function $H(x)$  defined by
	      \[
		      H(x) = \begin{cases}
			      0  \text{ if } x < 0 \\
			      1 \text { if } x \geq 0
		      \end{cases}
	      \]
	      is an important function in circuits and control theory
	      and is used to distinguish between an on and off state.
		  In this question, we will experiment with approximating $H$ via a polynomial.
	      \begin{enumerate}
		      \item \textbf{Approximate by Taylor Polynomials}
		            \begin{enumerate}
						\item Find a formula for the $n$th degree Taylor approximation to $H$ centered 
						at $1$. Does this provide a good approximation of $H$? If you changed where your
						Taylor approximation was centered, would your approximation improve?
			            \item The logistic function $L_k(x) = \frac{1}{1+e^{-2kx}}$ with scaling parameter $k$ 
						can be use to approximate the step function for large $k$.
						
						Find a third order Taylor polynomial $P_3(x)$ for $L_k(x)$ centered at $x=0$. 
						
						For this part, you may use the following facts without proof: $L_k'(0)=k/2$, $L_k''(0) =0$, and $L_k'''(0) = -k^3$.
			            \item What is the largest error between 
						$L_k$ and $P_3$ over the region $[-1,1]$? (Hint: Use Desmos to explore the error graphically.)

						\item In a particular electronics application $H(v)$ models the output 
						voltage of a circuit given input voltage $v$. You would like to model 
						this situation with a continuous function, $f$, subject to the following requirement:
						$f(v)$ is within 1\% of the value $1$ volt when $v\geq .01$ volts.

						Find a value of $k$ so that $L_k$ approximates $h$ within the required tolerance.
						Then, find a Taylor approximation of $L_k$ that meets the requirement.
		            \end{enumerate}


		      \item \textbf{Approximation by Polynomial Interpolation}

		            \begin{enumerate}
			            \item Pick four interpolation points $x_1,x_2,x_3,x_4 \in [-1,1]$, and construct a third order polynomial, $Q_3$, interpolating $H(x)$ using Lagrange's method.
			            \item Graph $H$, $Q_3$, and $P_3$ (your 3rd-order Taylor approximation from the previous part) in Desmos.
						Do both approximations do a good job? When would you use one approximation over the other?
		            \end{enumerate}

	      \end{enumerate}
    \clearpage
	\item The functions $\sin$ and $\cos$ are closely linked to $e^x$ by Euler's formula: $e^{i\theta}=\cos\theta+i\sin\theta$.
	      \begin{enumerate}
		      \item Find an expression for $\sin\theta$ as a linear combination of exponentials (Hint: 
			  think about $e^{i\theta}$ and $e^{-i\theta}$).
			  \item Find an expression for $\cos\theta$ as a linear combination of exponentials.
			  \item The \emph{hyperbolic sine} can be define from $\sin$ by the formula $\sinh\theta =-i\sin(i\theta)$. 
			  Use this fact to find a formula for $\sinh\theta$ in terms of exponentials.
			  \item The \emph{hyperbolic cosine} can be define from $\cos$ by the formula $\cosh\theta =\cos(i\theta)$.
			  Use this fact to find a formula for $\cosh\theta$ in terms of exponentials.
			  \item $\sin$ and $\cos$ satisfy the identity $\sin^2\theta+\cos^2\theta=1$. Find a similar identity
			  for $\sinh$ and $\cosh$.
	      \end{enumerate}
\end{enumerate}
