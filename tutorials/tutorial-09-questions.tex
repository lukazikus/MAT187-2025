\begin{objectives}
	In this tutorial you will practise solving 2nd order homogeneous ODEs.
\end{objectives}

\vspace{-.5em}
\subsection*{Problems}
\vspace{-.5em}




%%%%%%%%%%%%%%%%%%%%%%%%%%


\begin{enumerate}
	\item Consider two differentiable, non-constant functions $y_1(t)$ and $y_2(t)$ such that:
    \begin{enumerate}[label=(\roman*)]
        \item $y_1(t) \rightarrow +\infty$ as $t \rightarrow +\infty$
        \item $y_2(t)$ is periodic.
        \item $y_2(0) = \pi$.
    \end{enumerate}

    Is it possible that both $y_1(t)$ and $y_2(t)$ solve the \textit{same} constant-coefficient ODE $ay'' + by' + cy = 0$? Explain.

    \item You're going to build a clock: an oscillator that approximately counts out short periods of time.
    You're given a spring that has damping coefficient $b=2$ kg/s and spring constant $k=2$ kg/s$^2$, and a $4$ kg mass. The corresponding ODE for this problem is given by:
    
    \[
        4y'' + 2y' + 2y = 0
    \]
    
    What is the period of the trigonometric functions found in the resulting general solution? This number is the \textbf{quasi-period}, and for small values of $b$ it approximates well the actual time between peaks of the motion. Why is this just a quasi-period and not an actual period?

    \item 
    Consider the temperature of two adjacent rooms in a house. Room $A$ has no windows, and room $B$ has windows. Denote by $A$ the temperature in room $A$ and by $B$ the temperature in room $B$. We measure temperature in degrees Celsius and time in hours.
    
    There are three effects affecting the temperature [units are given in brackets for reference]:
    \begin{itemize}[nosep,itemsep=2mm,topsep=2mm]
    	\item Effect 1: The temperature in room $A$ increases/decreases at a rate proportional to the difference in temperature between room $A$ and room $B$. The proportionality constant is $2\left[\frac{1}{h}\right]$. Note that if room $B$ is warmer than room $A$, this means that this effect warms up room $A$.
    	\item Effect 2: The temperature in room $B$ increases/decreases at a rate proportional to the difference in temperature between room $B$ and room $A$. The proportionality constant is $2\left[\frac{1}{h}\right]$. Note that if room $A$ is warmer than room $B$, this means that this effect warms up room $B$.
    	\item Effect 3: Since Room $B$ has windows, the outside temperature (day/night) affects the temperature in room $B$. This effect leads to an additional change in temperature in room $B$ of $\frac12\sin \left(\frac{\pi t}{12}\right)$ $\left[\frac{\degree C}{h}\right]$.
    \end{itemize}

\begin{enumerate}
	\item Set up an ODE describing the change in temperature in room $A$. Explain.
    \item Set up an ODE describing the change in temperature in room $B$. Explain.
    \item Manipulate the two ODEs and combine them to get an ODE that does NOT include $B$ or its derivatives. In other words, eliminate $B$ completely from the equation.
    \item Looking at the ODE you found in part (c), relate features of the ODE to the applied context. Here are some questions to consider:
	\begin{itemize}[nosep]
		\item Is this oscillator undamped, overdamped, underdamped, or criticially damped?
		\item Is this oscillator free or forced?
		\item Looking \textit{at the applied context}, why do these properties make sense?
	\end{itemize}
\end{enumerate}

\begin{minipage}{12cm}
\item The problem on the right is called a Boundary Value Problem (BVP) -- instead of an Initial Value Problem -- since there is a value given at the starting time $t=0$ and at an end time $t=t_1$.\medskip

Find a value for $t_1$ such that this BVP has a solution (there are many correct choices for $t_1$!)
\end{minipage}
\hfill
\begin{minipage}{5cm}
\[
\begin{cases}
y'' + 4y' + 5y = 0\\
y(0) = 1\\
y(t_1) = 0
\end{cases}
\]
\end{minipage}

\end{enumerate}