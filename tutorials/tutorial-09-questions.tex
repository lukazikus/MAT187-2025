\begin{objectives}
	In this tutorial you will practise solving 2nd order homogeneous ODEs.
\end{objectives}

\vspace{-.5em}
\subsection*{Problems}
\vspace{-.5em}




%%%%%%%%%%%%%%%%%%%%%%%%%%


\begin{enumerate}
	% \item Consider two differentiable, non-constant functions $y_1(t)$ and $y_2(t)$ such that:
    % \begin{enumerate}[label=(\roman*)]
    %     \item $y_1(t) \rightarrow +\infty$ as $t \rightarrow +\infty$
    %     \item $y_2(t)$ is periodic.
    %     \item $y_2(0) = \pi$.
    % \end{enumerate}

    % Is it possible that both $y_1(t)$ and $y_2(t)$ solve the \textit{same} constant-coefficient ODE $ay'' + by' + cy = 0$? Explain.

    \item You're going to build a clock: an oscillator that approximately counts out short periods of time.
    You're given a spring that has damping coefficient $b=3$ kg/s and spring constant $k=2$ kg/s$^2$, and a $4$ kg mass. The corresponding ODE for this problem is given by
    \[
        4y'' + 3y' + 2y = 0
    \]
    where $y(t)$ is the amount that the spring is stretched  at time $t$ relative to it's neutral position (i.e., it's stretch at rest).

    \begin{enumerate}
        \item By doing a unit-based analysis, explain whether or not 
        \[
            4y''+2y'+3y=0
        \]
        could also be a model for the dampened spring.

        \item Find the characteristic equation for the ODE, but don't solve it.

        \item Without solving the characteristic equation: do you expect it to have all real roots or a some complex roots? Explain.

        Compute the roots. Did this match your expectation?

        \item Write down the general, complex, solution to the ODE.

        \item Write down the general, real, solution to the ODE.

        \item This dampened spring model has a quasi-period of XXXXXX. Why
        is XXXXXX called a quasi-period and not a period?
    \end{enumerate}
    

    \item 
    Consider the temperature of two adjacent rooms in a house. Room $A$ has no windows, and room $B$ has windows. Denote by $A(t)$ the temperature in room $A$ at time $t$ and by $B(t)$ the temperature in room $B$ at time $t$. We measure temperature in degrees Celsius and time in hours.
    
    There are three effects affecting the temperature [units are given in brackets for reference]:
    \begin{itemize}[nosep,itemsep=2mm,topsep=2mm]
    	\item Effect 1: The temperature in room $A$ increases/decreases at a rate proportional to the difference in temperature between room $A$ and room $B$. The proportionality constant is $2\left[\frac{1}{h}\right]$. Note that if room $B$ is warmer than room $A$, this means that this effect warms up room $A$.
    	\item Effect 2: The temperature in room $B$ increases/decreases at a rate proportional to the difference in temperature between room $B$ and room $A$. The proportionality constant is $2\left[\frac{1}{h}\right]$. Note that if room $A$ is warmer than room $B$, this means that this effect warms up room $B$.
    	\item Effect 3: Since Room $B$ has windows, the outside temperature (day/night) affects the temperature in room $B$. This effect leads to an additional change in temperature in room $B$ of $\frac12\sin \left(\frac{\pi t}{12}\right)$ $\left[\frac{\degree C}{h}\right]$.
    \end{itemize}

\begin{enumerate}
	\item Set up an ODE describing the change in temperature in room $A$.
     Label each term in your ODE that corresponds to an effect.
    \item Set up an ODE describing the change in temperature in room $B$.
     Label each term in your ODE that corresponds to an effect.
    \item Manipulate the two ODEs and combine them to get an ODE that does NOT include $B$ or its derivatives. In other words, eliminate $B$ completely from the equation.

    \item There are various models of a spring:
    \begin{itemize}
        \item undamped
        \item dampened
        \item underdamped
        \item overdamped
    \end{itemize}
    
    What type of spring model most closely resembles your model for the temperature of room $A$? Does this make physical sense? Explain.
    
\end{enumerate}

% \begin{minipage}{12cm}
% \item The problem on the right is called a Boundary Value Problem (BVP) -- instead of an Initial Value Problem -- since there is a value given at the starting time $t=0$ and at an end time $t=t_1$.\medskip

% Find a value for $t_1$ such that this BVP has a solution (there are many correct choices for $t_1$!)
% \end{minipage}
% \hfill
% \begin{minipage}{5cm}
% \[
% \begin{cases}
% y'' + 4y' + 5y = 0\\
% y(0) = 1\\
% y(t_1) = 0
% \end{cases}
% \]
% \end{minipage}

\end{enumerate}