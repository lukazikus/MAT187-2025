\subsection*{Learning Objectives}
	Students need to be able to\ldots
	\begin{itemize}
		\item Practise integration techniques: integration by parts, trigonometric integrals, trigonometric substitution
	\end{itemize}

\subsection*{Notes}
	\begin{enumerate}
			\item This should be a relatively straightforward application of integration by parts. Part B may be tricky for students since there is only one function in the integrand.

            \item This question may trip students up since it requires them to think about how they can use a table of values to apply the integration by parts formula. Try to encourage students to write out the formula and then see if they can use the table of values to aid in evaluating the integral.

            \item All of these parts are meant to have students practise trigonometric substitution. The first two parts should be relatively straightforward, but the last one is very involved. Note that it is possible to solve it with geometry, but encourage students to try solving it using an integral for the practice.

            % \item They are not likely to get to this question. This is meant to be an interesting application of integration by parts to show that the Gamma function satisfies an important property of the factorial function, and can extend it to other values. The actual integration by parts step itself isn't difficult, but there's a bit of a complication because they don't know integration by parts for infinite integrals yet, so things need to be phrased in terms of limits. 
	\end{enumerate}
